% Created 2010-06-13 Sun 22:33
\documentclass[11pt]{article}
\usepackage[utf8]{inputenc}
\usepackage[T1]{fontenc}
\usepackage{fixltx2e}
\usepackage{graphicx}
\usepackage{longtable}
\usepackage{float}
\usepackage{wrapfig}
\usepackage{soul}
\usepackage{t1enc}
\usepackage{textcomp}
\usepackage{marvosym}
\usepackage{wasysym}
\usepackage{latexsym}
\usepackage{amssymb}
\usepackage{hyperref}
\tolerance=1000
\documentclass[a4paper,norsk,10pt]{article}
\usepackage{fullpage} %Utnytter større del av arket.
\usepackage[sc,osf]{mathpazo} %Pen font.
\usepackage[norsk]{babel} %Norsk språkpakke.
\usepackage{hyperref} %For lenker i latex-dokument.
\hypersetup{colorlinks=true,linkcolor=black} %Farge på lenker.
\usepackage{amsmath} %Større matematikkpakke.
\usepackage[usenames,dvipsnames]{color} %Flere farger.
\usepackage{parskip} %Norsk type linjeskift.
\providecommand{\alert}[1]{\textbf{#1}}

\title{INF1010 - Eksamensnotater}
\author{Eivind, Sjur, Annabelle \& Martin}
\date{13 June 2010}

\begin{document}

\maketitle

\setcounter{tocdepth}{3}
\tableofcontents
\vspace*{1cm}
\section{Tall}
\label{sec-1}



\begin{center}
\begin{tabular}{ll}
\hline
 \textbf{Type}  &  \textbf{Lovlige verdier}          \\
\hline
 byte           &  -+127                             \\
\hline
 short          &  -+32 767                          \\
\hline
 int            &  -+2 147 483 647                   \\
\hline
 long           &  -+9 223 372 036 854 775 807L      \\
\hline
 float          &  -+3.402 823 47 E+38F              \\
\hline
 double         &  -+1.797 693 134 862 315 70 E+308  \\
\hline
\end{tabular}
\end{center}



  negative tall er har alltid èn større range pga 0. 
\subsection{Integere}
\label{sec-1_1}


   Integere er heltall, og lite annet.
   Hvis vi vil konvertere en streng til en int sier vi:
\begin{verbatim}
int x = Integer.parseInt("123");
\end{verbatim}
   Hvis vi vil konvertere(caste) et annet tall til int skriver vi:
\begin{verbatim}
int x = (int) 3.14
\end{verbatim}


\begin{center}
\begin{tabular}{lll}
\hline
 Operasjon            &  Beskrivelse          &  Eksempel             \\
\hline
 +, -, *, /           &  De fire regneartene  &  1+1                  \\
\hline
 ++                   &  legge til 1          &  i++                  \\
\hline
 --                   &  trekke fra 1         &  i--                  \\
\hline
 \%                   &  modulo(rest)         &  5\%2 -> 1            \\
\hline
 Math.sqrt(\ldots{})  &  Kvadratrot           &  Math.sqrt(4) -> 2    \\
\hline
 Math.pow(x, y)       &  Potens (x$^y$)       &  Math.pow(2, 3) -> 8  \\
\hline
\end{tabular}
\end{center}


  
   
\begin{itemize}
\item Metodekall

\begin{itemize}
\item ++ og --
\item * og /
\item + og -
\item -(som negativt fortegn)
\end{itemize}

\end{itemize}


   husk at:
\begin{verbatim}
int x, y=1;
x = y++ + y + ++y;
\end{verbatim}
   nå er x = 6, fordi stykket blir 1 + 2 + 3. først er y = 1, og 
   så øker den slik at neste gang y brukes er den 2. 
   når vi skriver ++y så vil y være 3 før vi legger det til.
\subsection{desimaltall}
\label{sec-1_2}


   Jeg velger å bare leke med floats, men reglene er ca det samme som for int.

   konvertering fra heltall til desimaltall:
\begin{verbatim}
float x = (3 + 0.0) / 2;
\end{verbatim}

   altså vi bare legger til 0.0 som det som skjer først, slik at ett av tallene er
   et desimaltall, ellers vil vi få heltallsdivisjon. Enkelt og greit,
   utenom det gjelder de samme operatorene og de samme presidensreglene.
\section{Sannhetsverdier (boolean)}
\label{sec-2}


\begin{itemize}
\item \&\&

\begin{itemize}
\item Og
\item true \&\& true
\item ||

\begin{itemize}
\item Eller
\item true || false
\end{itemize}

\end{itemize}

\item ikke
       b = !false
\item < og >

\begin{itemize}
\item mindre enn, større enn
\item b = x<y
\end{itemize}

\item <= og >=

\begin{itemize}
\item mindre eller lik
\item b = x<=y
\end{itemize}

\item ==

\begin{itemize}
\item er lik
\item b = x==y
\end{itemize}

\item !=

\begin{itemize}
\item er ikke lik
\item b = x!=y
\end{itemize}

\end{itemize}
\subsection{presidensregler}
\label{sec-2_1}


\begin{itemize}
\item Metodekall

\begin{itemize}
\item !
\item <, <=, >=, >
\item \texttt{=, !}
\item \&\&
\item ||
\end{itemize}

\end{itemize}
\section{Tekst - Strenger og char}
\label{sec-3}


  vi har strengen s = ``kake''
  

\begin{center}
\begin{tabular}{lll}
\hline
 Navn                  &  Forklaring                           &  Eksempel                  \\
\hline
 charAt(\ldots{})      &  tegnet i gitt posisjon(fra 0)        &  s.charAt(2)=='k'          \\
\hline
 length()              &  gir lengden på teksten               &  s.length()==4             \\
\hline
 substring(\ldots{})   &  delteksten fra- og tilposisjon       &  s.substring(1,3)==''ak''  \\
                       &  gir indeksen og ut                   &  s.substring(1)==''ake''   \\
\hline
 equals(\ldots{})      &  tester likhet mellom strenger        &  s.equals(``kake'')        \\
                       &  (boolean)                            &                            \\
\hline
 indexOf(\ldots{})     &  posisjonen til tegnet/tekst          &  s.indexOf(`a')==1         \\
\hline
 startsWith(\ldots{})  &  starter teksten med \ldots{} (bool)  &  s.startsWith(``ka'')      \\
\hline
 endsWith(\ldots{})    &  ender teksten med \ldots{} (bool)    &  s.endsWith(``ke'')        \\
\hline
 compareTo(\ldots{})   &  sammenligning av tekster             &  s.compareTo(``bake'')<0   \\
\hline
 toCharArray()         &  gjør om strengen til et array        &  s.toCharArray()           \\
                       &  av chars                             &                            \\
\hline
\end{tabular}
\end{center}
\subsection{char}
\label{sec-3_1}


   en char-verdi er rett og slett en bokstav, den kan sammenlignes (`a' < `b') 
   og vil da sammenlignes ut
   i fra ascii-verdier (alle store bokstaver er mindre enn de små bokstavene).
   og kan dermed i mange sammenhenger tenkes på som tall.
\subsection{String}
\label{sec-3_2}


   En string er en rekke med char-verdier, altså ord. Man kan legge ord
   sammen med pluss-opperatoren 
   (``heisann'' + `` `` + navn), man kan konvertere tall til strenger på denne måten
\begin{verbatim}
String s = "" + 42;
\end{verbatim}
   og verdien til s vil være ``42''.

   Strenger kan også deles opp i arrayer ved hjelp av en split-funksjon. eks:
\begin{verbatim}
string[ ] t = s.split(" ");
\end{verbatim}
\subsection{Innlesing av tekst}
\label{sec-3_3}

for å lese fra terminal er det lettest å bruke en klasse som
finnes i og java.util.*;

Vi bruker også System.in

\begin{verbatim}
import java.util.*;
Scanner tast = new Scanner(System.in);

String s = tast.nextLine();
\end{verbatim}

Fra fil så trenger vi også noen klasser i java.io.*
`programmet' under leser fra en fil og over i terminalen linje for linje.

\begin{verbatim}
import java.util.*;
import java.io.*;

Scanner f = new Scanner(new File(filnavn));

while (f.hasNextLine()){
  String inn = f.nextLine();
  System.out.println(inn);
}
\end{verbatim}
\section{Arrayer}
\label{sec-4}


  Arrayer er en indeksert(fra 0) gruppe av objekter. Man må definere
  størelsen når man lager objektet.
\begin{verbatim}
String[] a = new string[3];
\end{verbatim}
  man kan nå finne lengden på arrayet og bruke det som en int a.length;

  Man kan lage arrayer av alle mulige objekter.
\subsection{array-metoder}
\label{sec-4_1}
\subsubsection{rask sortering}
\label{sec-4_1_1}

\begin{verbatim}
import java.util.*;
char[] sorteres = "noesomskalsorteres".toCharArray();
Arrays.sort(sorteres);
for(int i = 0;i<sorteres.length;i++){
    System.out.print(sorteres[i]+" ");
}
\end{verbatim}

Arrays er en klasse i java.util som kan sortere for oss ``raskt og gæli''
bruker antagelig quicksort under panseret og funker med de fleste tall og 
chars.
\section{Løkker og kontrollstrukturer}
\label{sec-5}



\begin{center}
\begin{tabular}{lll}
 \textbf{Navn}  &  \textbf{Beskrivelse}                 &  \textbf{Eksempel}                             \\
 for            &  bestemt antall ganger                &  for(int i=0;i<3;i++)\{\}                      \\
                &  alle objekter i array                &  for(String s : a)\{\}                         \\
                &  alle objekter i hash                 &  for(String s : hm.values())                   \\
 while          &  i mens test er sann                  &  while(b)\{\}                                  \\
 do-while       &  utfører løkka før testen             &  do \{\} while(b);                             \\
 switch         &  Hopper mellom kodeblokker ved input  &  switch(c)\{case a: <>; break; default: <>;\}  \\
\end{tabular}
\end{center}
\subsection{while}
\label{sec-5_1}

   
   while-løkker er kanskje den enkleste formen for løkker, den gjør en blokk kode
   så lenge predikatet i parameteret er sant. 

   Man trenger ikke nødvendigvis å kjøre
   en kodeblokk, for eksempel hvis man har en boolsk funksjon (``kan()'') som parameteret
   kan man kjøre koden ``while(kan());'' og dermed si at man skal kjøre helt til ``kan()''
   returnerer false.
\subsection{for}
\label{sec-5_2}

   
   for-løkker er en naturlig utvidelse av while da man ofte trenger tellere eller ting
   som skjer for hver gang kodeblokken skal kjøres, for eksempel en teller når man går
   igjennom et array.

\begin{verbatim}
for(int i=0;i<array.length;i++;){
  System.out.println(array[i])
}
\end{verbatim}

   Men for-løkker kan brukes på andre måter da den har en ganske enkel måte å oppføre 
   seg på ``for(initialiseringskode;predikat;postkode)'', eneste som det er strengt hva 
   man må ha i en for-løkke er predikatet. initialiseringen, kan være å sette en teller
   til noe, eller når man går igjennom noder i en graf, så kan man f.eks sette 
   src$_{\mathrm{java}}$\{for(Node n = root;n.next!=null;n=n.next)\}
\subsection{while .. do}
\label{sec-5_3}

   
   Mindre brukt løkketype, egentlig en while-løkke hvor koden kjøres minst en gang før
   predikatet blir testet.
  
\subsection{switch}
\label{sec-5_4}


Ved bruk av if-setning kan man velge en av to muligheter ved at
setningen som evalueres er \emph{true} eller \emph{false}. Med \emph{switch} kan man
velge mellom mange inputpåstander som er \emph{integers}, \emph{chars} eller \emph{enums}. 

\begin{verbatim}
switch (<utrrykk>) {
    case verdi1: <settninger>; break;
    etc..
    default: <kjøres ved ingen treff ved <uttrykk> >;
}
\end{verbatim}
\section{Hashmaps}
\label{sec-6}


  Hashmaps er en enkel måte å ordne mange objekter med et objekt som
  indeks.\\
  import java.util.*\\
  HashMap<string,Person> personregister = new HashMap<String,Person>();\\
  

\begin{center}
\begin{tabular}{ll}
\hline
 Metode                 &  beskrivelse                                  \\
\hline
 put(nøkkel, peker)     &  legge til objekt i HM                        \\
\hline
 get(nøkkel)            &  hente peker til objekt                       \\
\hline
 remove(nøkkel)         &  fjerne nøkkel fra HM                         \\
\hline
 containsKey(nøkkel)    &  bool om nøkkelen er der                      \\
\hline
 containsValue(objekt)  &  bool om objektet er der                      \\
\hline
 values()               &  lager en mengde av alle verdiene i HM,       \\
                        &  brukes til itterering                        \\
\hline
 keySet()               &  brukes til å lage en mengde av alle nøklene  \\
                        &  brukes til iterering                         \\
\hline
 isEmpty()              &  returnerer true hvis tabellen er tom.        \\
\hline
 size()                 &  Metoden returnerer antall nøkler i tabellen  \\
\hline
\end{tabular}
\end{center}



Hashmaps
import java.util.*;

HashMap hashmapnavn = new HashMap();
hashmapnavn.put(nøkkel, verdi);

Hente: Bil b = (Bil) register.get(nøkkel);

Hente alle verdiene:
Iterator it = register.values().iterator();
while (it.hasNext()) \{
    Bil b = (Bil) it.next();
\}


containsKey(nøkkel) sjekker om objekt med nøkkelen finnes. Gir
true/false.
containsValue(verdi) sjekker om hashmap inneholder gitte objekt/verdi.
size() antall nøkler.
\section{UML}
\label{sec-7}


Diagrammer av programmene vårt.
Tegner selve strukturen til programmet.

-Objekdiagrammer. (UML-bokser med ev. variabler og argumenter etc.)
-Klassediagrammer. (bare koblingene etc.)

Tegner selve arkitekturen til programmet vårt. 
\subsection{Navneklassen}
\label{sec-7_1}


+public
-private
\~{}package

Vi lager en modell av problemområdet vårt (også kalt domenemodell).
-Modell av probleme (og modell av databasen).

-Tegner streker mellom klassene. Og navn på forbindelsen.

Vi tegner hvor mange objekter det maksimalt kan være.

Vi har forskjellige skrivemåter for verdier:
1 - en
 * -  null, en eller flere
1..* - minst en
3,4,5 - tre fire eller fem

Tegner koblinger mellom de klassene du tenker programmet skal snakke
med.

Vi angir kun de mest sentrale dataene i UML-boksene.. F.eks. viktige
variabler og arrayer.
F.eks. :klasenavn [atributt-fekt (kan være tomt)].

-Pekere - piler

Tegner flere ved å legge en UML-boks som ``skygge'' bak.. En
representasjon for N antall.
\section{Lenkede lister}
\label{sec-8}
\subsection{Hva er en lenket liste?}
\label{sec-8_1}

En lenket liste er en liste med objekter refererer til neste, eller forrige objekt ved hjelp av pekere.

-Lenket liste. Trenger ikke ta stilling til antall fra start.
-Alle objekter er lenket til hverandre, eget objekt som peker på
-første objektet.
-Må søke i listen forfra for å finne objekter.


\begin{verbatim}
/* En første skisse av personlista. Først klassen som
 * beskriver objektene (personene) som skal lenkes sammen.
 * Deretter en klasse som beskriver selve lista. */

class Person {
    String navn;
    Person nestePerson ;
    // andre attributter
}

class Personer {
    Person personliste ;
    // . . .
}
// Lage en ny liste med personer:
Personer mineVenner = new Personer ;
\end{verbatim}

Hvilke operasjoner trenger vi?
\begin{itemize}
\item Sette inn ny person
\item Finne en person
\item fjerne en person
\end{itemize}

\ldots{}

\begin{verbatim}
class Personer {
    Person personliste ;
    void setteInnPerson ( Person inn ) { }
    void finnePerson ( Person p ) { } // Finne person og taUtPerson
    void taUtPerson ( Person ut ) { } // Kan også slås sammen om hvert 
                                 // objekt bare skal brukes èn gang
    // . . .
}
\end{verbatim}
\subsection{Innsettingsmetoder}
\label{sec-8_2}
\subsubsection{LIFO}
\label{sec-8_2_1}

Innsetting først (LIFO):

\begin{verbatim}
void settInnPersonForst ( Person inn ) {
    // Hvis lista er tom, sett inn objektet
    if(personliste == null) personliste = inn ;
    else {
        // minst et objekt i lista
        inn.nestePerson = personliste ;
        personliste = inn;
    }
}
\end{verbatim}
\subsubsection{FIFO}
\label{sec-8_2_2}

Innsetting sist (FIFO):

CANNOT INCLUDE FILE code/lenkeeksempel4.java
\subsubsection{Innsetting etter et bestemt objekt}
\label{sec-8_2_3}


\begin{verbatim}
/** Setter personobjektet i inn etter personobjektet e
* Hverken i eller e er NULL.
* @param e Personobjektet som skal ha i som neste
* @param i Personobjektet som skal inn etter i
*/

void settInnPersonEtter ( Person e , Person i ) {
    i.nestePerson = e.nestePerson ;
    e.nestePerson = i ;
}
\end{verbatim}
\subsection{Skrive ut en lenket liste}
\label{sec-8_3}


\begin{verbatim}
// I klassen Personer :
void skrivAlle() {
    Person p = personliste ;
    while (p != null ) {
        p.skrivUt() ;
        p = p. nestePerson ;
    }
}
// I klassen Person :
void skriv ( ) {
System. out . println ( . . . ) ;
}
\end{verbatim}
\section{Rekursjon}
\label{sec-9}


Kaller seg selv.
Midlertidig data lagres i en såkalt Call Stack. Når metode kommer til
return statement går man ``baklengs'' tilbake i call stacken og metoden
fortsetter etter der den kalte seg selv.

Klassisk eksempel på en rekursjonsmetode er dette fakultetseksempelet.

\begin{verbatim}
class Factorial {
     int fact(int n) {
         if ( n ==1) return 1;
         return fact (n-1) * n;
     }
}

class Recursion {
     public static void main (String args[]) {
          Factorial f =new Factorial();
          System.out.println(?Factorial of 3 is ? + f.fact(3));
          System.out.println(?Factorial of 4 is ? + f.fact(4));
          System.out.println(?Factorial of 5 is ? + f.fact(5));
     }
}
\end{verbatim}

Dersom fact() blir kalt med tallet 1, vil den returnere 1, ellers vil den returnere produktet av fact(n-1)*n. Denne metoden rekurserer inntil n = 1.
Når en metode kaller seg selv, vil nye lokale variable og parametere lagres i stack, og koden utføres med disse nye variablene og parameterene fra starten aav. Den lagerr ikke kopi av metoden, kun argumenter/variabler.
\subsection{Hanois tårn}
\label{sec-9_1}
\section{Binære trær}
\label{sec-10}
\subsection{Sett inn i binærtre}
\label{sec-10_1}

Under følger et eksempel på en sett-inn-metode for binære
trær. Metoden ligger i klassen \emph{BinaryTreeNode}, og vi antar at vi har
en metode som sammenligner verdiene til nodene, kalt \emph{compareNodes}.

\begin{verbatim}
void addNodeToBinaryTree(Node node) {
    if(compareNodes(node) <= 0) {
        if(right == null) {
            right = node;
        } else {
            right.addNodeToBinaryTree(node);
        }
    } else {
        if(right == null) {
            left = node;
        } else {
            left.addNodeToBinaryTree(node);
        } 
    }
}
\end{verbatim}
\subsection{Traversering av treet}
\label{sec-10_2}
\subsubsection{Preorder}
\label{sec-10_2_1}
\subsubsection{Inorder}
\label{sec-10_2_2}

Hvis vi ønsker å traverse binærtreet i synkende rekkefølge bruker vi
\emph{inorder traversal}.

\begin{verbatim}
void printNodes() {
   if(left != null) {
       left.printNodes();
   }
   System.out.print(node.value);
   if (right != null) {
       right.printNodes();
   }
}
\end{verbatim}
\subsubsection{Postorder}
\label{sec-10_2_3}
\section{Søk}
\label{sec-11}

  hvis vi har sorterte data vil alltid binærsøk være den raskeste måten
  å søke på. Binærsøk så halverer vi antall muligheter for hver et eksempel
  fra den virkelige verden er hvis man skal gjette et tall mellom 1 og 1024
  og man har 10 gjetteforsøk, hvor man får vite om man gjettet for stort eller
  for lite, vil man alltid kunne finne tallet ved at man hele tiden halverer.

  eks: vi skal finne tallet (769)

\begin{center}
\begin{tabular}{rrlr}
\hline
 mulige tall  &  gjett  &  større / mindre  &  forsøk  \\
\hline
        1024  &    512  &  større           &       1  \\
\hline
         512  &    768  &  større           &       2  \\
\hline
         256  &    896  &  mindre           &       3  \\
\hline
         128  &    832  &  mindre           &       4  \\
\hline
          64  &    800  &  mindre           &       5  \\
\hline
          32  &    784  &  mindre           &       6  \\
\hline
          16  &    776  &  mindre           &       7  \\
\hline
           8  &    772  &  mindre           &       8  \\
\hline
           4  &    770  &  mindre           &       9  \\
\hline
           2  &    769  &  riktig!          &      10  \\
\hline
\end{tabular}
\end{center}



  ved usorterte mengder er det vanskeligere å gjøre strukturerte søk og det kan ofte
  være lurt å enten sortere det først, eller dele opp mengden i delmengder og søke hver
  del for seg selv.
\subsection{Eksempel på søking i usortert mengde med tråder}
\label{sec-11_1}


\begin{verbatim}
public class FinnMinst {
    final int maxVerdiInt = Integer.MAX_VALUE;
    int [ ] tabell; // mengden vi skal søke igjennom
    Oversikt oversikt; // her trådene raporterer

    public static void main(String[ ] args) {
        new FinnMinst();
    }

    public FinnMinst(){
        // vi lager mengden vi skal søke igjennom med random ints
        tabell = new int[64000];
        for (int in = 0; in< 64000; in++) 
            tabell[in] = (int)Math.round(Math.random()* maxVerdiInt);

        //vi lager et overvåkningsobjekt som mottar info fra trådene
        oversikt = new Oversikt();
        for (int i = 0; i< 64; i++)

            // Lager 64 tråder, og hvilken del av arrayet de skal søke på
            new Soketraad(tabell,i*1000,((i+1)*1000)-1,oversikt).start(); 

        oversikt.vent(); // er ikke ferdig før alt er søkt igjennom
        System.out.println("Minste verdi var: " + oversikt.minste);
    }
}

class Oversikt {
    int minste = Integer.MAX_VALUE, antallFerdigeTrader = 0;  
    
    synchronized void vent() {
        while (antallFerdigeTrader != 64) {
            try {wait();} catch (InterruptedException e){}      
        }
    } 

    //mottar tall fra en tråd og ser om den er mindre enn den minste
    synchronized void giMinste (int minVerdi) {
        antallFerdigeTrader++;
        if(minste > minVerdi) minste = minVerdi;      
        notify();
    }
}

class Soketraad extends Thread {
    int [ ] tab; int start, end; Oversikt ov;
  
    Soketraad(int [ ] tb, int st, int en, Oversikt o) {
        tab = tb; start = st; end = en; ov = o; 
    }
 
    public void run(){
        int minVerdi = Integer.MAX_VALUE;     
        for (int i = start; ind <= end; i++)
            if(tab[i] < minVerdi) minVerdi = tab[i];
        ov.giMinste(minVerdi);
    } 
}
\end{verbatim}
\section{Sortering}
\label{sec-12}

  Det er ofte en fordel å sortere store mengder data for å gjøre
  søking i i data effektivt om i det hele tatt mulig.

  Vi har gjennomgått flere forskjellige måter å sortere data på
  jeg vil her fokusere på følgende
\begin{itemize}
\item innstikksortering
\item quicksort
\end{itemize}
\subsection{Innstikksortering}
\label{sec-12_1}

   veldig simpel sorteringsalgoritme som er effektiv på små mengder (i forhold til andre
   sorteringsalgoritmer), men taper mye når det er større mengder data som skal sorteres
   baserer seg på at man starter først i mengden og sorterer mens man itererer over mengden

\begin{verbatim}
class InSort{
    public static void main(String[] args){
        // mengden som skal sorteres 
        int[] mengde = {4, 2, 6, 3, 7, 1, 5}; 
        for(int i=0;i<mengde.length;i++){
            // hvor vi er og hva vi skal stikke inn 
            int j=i, temp=mengde[i]; 
            // går igjennom alt som er sortert frem til punktet der 
            // vi enten er på starten av lista, eller at det neste 
            // sorterte elementet er mindre enn temp
            while(j>0 && mengde[j-1] > temp){ 
                // Vi dytter lista fremover for å gjøre plass til innstikk
                mengde[j]=mengde[j-1]; 
                j--;
            }
            // når vi har funnet riktig sted setter vi inn
            mengde[j] = temp;
        }
    }
}
\end{verbatim}
\subsection{Quicksort}
\label{sec-12_2}


   Mer populær for mellomstore til store datamengder, den tar et objekt fra mengden og
   forterer listen etter hva som er større og mindre enn det objektet, deretter begynner
   den å sortere de to delene på samme måte rekursivt (dermed lettere å implementere
   tråder)

\begin{verbatim}
public class QSort {
    static int antallBytt = 0;

    public static void main(String[] args) {
        char [] tegn = "åøæzywvutsrqponmlkjihgfedcba".toCharArray();
        quicksort(tegn,0,tegn.length-1);
        for(int i=0;i<tegn.length;i++){System.out.print(tegn[i]+" "); }
    }
    static void quicksort(char[] a, int fra, int til) {
        //man kan ikke sortere bare en ting
        if (!(til-fra < 1)){
            //vi velger et tall som vi skal
            char pivot = a[til];
            boolean ikkeFerdig = true;
            int forst = fra;     // der vi midlertidig pivoterer rundt, 
            int sist = til - 1;  // snevrer oss mot et punkt
            while (ikkeFerdig) {
                while (a[forst] < pivot && forst<a.length-1) { forst++ ;}// vi finner stedet
                while (a[sist] > pivot && sist>0) { sist--;}// der vi vil bytte
                if (forst < sist) { // hvis vi ikke har funnet midten
                    bytt(a, forst, sist);
                    forst++; sist--;
                } else ikkeFerdig = false; 
            }
            bytt(a, forst, til);
            quicksort(a, fra, forst-1);
            quicksort(a, forst+1, til);
        }
    }
    static void bytt (char[] a, int fra, int til) {
        char temp = a[til];
        a[til] = a[fra];
        a[fra] = temp;
    }
}
\end{verbatim}
\subsection{Sortering med tråder}
\label{sec-12_3}

Raskt eksempel med Quicksort som bruker tråder.

\begin{verbatim}
public class QSortTrad {
    int antallTrader = 1, ferdige = 0;
    char[] a;

    QSortTrad(){
        a = "åøæzywvutsrqponmlkjihgfedcba".toCharArray();
        new Qtrad(0,a.length-1).start();
        while(antallTrader!=ferdige){
            try { Thread.sleep(0); } catch (Exception e) {};
        }
        for(int i=0;i<a.length;i++){System.out.print(a[i]+" ");}
    }
    
    public static void main(String[] args) {new QSortTrad();}

    class Qtrad extends Thread {
        int fra, til;
        Qtrad(int i, int j){fra = i; til = j;}

        public void run() {
            if (!(til-fra < 1)){
                char pivot = a[til];
                boolean ikkeFerdig = true;
                int forst = fra; 
                int sist = til - 1;

                while (ikkeFerdig) {
                    while (a[forst] < pivot && forst<a.length-1) { forst++ ;}
                    while (a[sist] > pivot && sist>0) { sist--;}
                    if (forst < sist) { 
                        bytt(a, forst, sist);
                        forst++; sist--;
                    } else ikkeFerdig = false; 
                }
        
                bytt(a, forst, til);
                antallTrader++;
                new Qtrad(fra, forst-1).start();
                fra = forst+1;
                run();
            } else { 
                ferdige++;
            }
        }
        void bytt (char[] a, int fra, int til) {
            char temp = a[til];
            a[til] = a[fra];
            a[fra] = temp;
        }
    }
}
\end{verbatim}
\section{Packages, interfaces, tilgangsnivåer og abstract}
\label{sec-13}
\subsection{Packages}
\label{sec-13_1}

Pakker er grupper med relaterte klasser, som hjelper deg med å 
organisere koden din. Klasser definert i en pakke, er kun tilgjengelig
via pakkenavnet. Med pakker har man også muligheten til å gjøre
klasser kun tilgjengelige fra innsiden av pakken, noe som kan gi økt
kontroll og sikkerhet. 

En annen stor fordel med pakker er at når du navngir en pakke,
opretter du et nytt \emph{namespace}. Dette får man veldig bruk for når man
jobber på større prosjekter. To klasser i Java kan ikke ha samme navn,
for da får vi navnkollisjon. Men med namespaces løses dette. Det gjør
at vi kan ha to klasser med samme navn, men som ligger i forskjellige
pakker, eller namespaces, slik: \emph{package01.CommonName.java} og
\emph{package02.CommonName.java}. 

Alle klasser i Java hører til en pakke, men når ingen blir spesifisert
blir default (global package) benyttet. Denne pakken har heller ikke
noe navn.
\subsubsection{Opprette en pakke}
\label{sec-13_1_1}

For å oprette en pakke bruker vi følgende syntax:

\begin{verbatim}
package MyPackage01;
\end{verbatim}

Selve filene i pakken må da legges i en mappe med nøyaktig samme navn,
\emph{\/MyPackage01\/}. Man kan også ha et hierarki av pakker, f.eks. 

\begin{verbatim}
package MyPackage01.MyPackage02...
\end{verbatim}

Og vi definerer mappestrukturen deretter. Det kan være lurt å følge
Maven-mappestrukturen, da det er en godt organisert og veletablert
standard. 

For at Java skal finne pakkenne må de enten ligge i samme mappe som du
allerede jobber i, eller så kan man manuelt sett \textbf{CLASSPATH} som Java
leter etter.
\subsubsection{Importering av pakker}
\label{sec-13_1_2}

Istedenfor å extende hver klasse i en pakke, for å kunne benytte
klassene kan man bruke \emph{import}, med eller uten wildcard, slik:

\begin{verbatim}
import Package.Class // Importerer spesifikk klasse.
import Package.* // Importerer alle klasser i pakken.
\end{verbatim}

Hele Java API'et er definert i pakker og importeres på samme måte. Det
eneste er \emph{java.lang}, som automatisk importeres, som blant annet
inneholder \emph{System}-klassen og \emph{println()}.
\subsection{Interfaces}
\label{sec-13_2}

Bruker interface på en klasse hvis vi ønsker egenskaper som flere
forskjellige klasser skal arve. Vi beskrivker kun ønsket oppførsel, og
ikke hvordan det skal gjøres. Metoder må skrives på nytt for hver
gang det implementeres. 

Et inteface defineres slik:

\begin{verbatim}
access interface InterfaceName {
    returntype methodName(parameter);
    type varName = value;
}
\end{verbatim}

Og implementeres med, 

\begin{verbatim}
class ClassName implements InterfaceName{ }
\end{verbatim}

Vi kan implementere så mange interfaces vi ønsker, og separerer dem
med komma.
\subsubsection{Variabler i et interface}
\label{sec-13_2_1}


Variabler i et interface vil implisitt være \emph{public}, \emph{static} og
\emph{final}. Det gjør det veldig lett å dele variabler over flere klasser
i et program, ved å implementere dette interfacet.
\subsubsection{Interfaces kan extende}
\label{sec-13_2_2}


Vi kan også la et interface extende et annet interface, og hvis en
klasse implementerer det ene interfacet, må den også implementere alle
metoder som er arvet fra det andre interfacet.
\subsection{Tilgangsnivåer}
\label{sec-13_3}


\begin{center}
\begin{tabular}{lllll}
 \textbf{Modifier}   &  \textbf{Class}  &  \textbf{Package}  &  \textbf{Subclass}  &  \textbf{World}  \\
 \emph{public}       &  Y               &  Y                 &  Y                  &  Y               \\
 \emph{protected}    &  Y               &  Y                 &  Y                  &  N               \\
 \emph{no modifier}  &  Y               &  Y                 &  N                  &  N               \\
 \emph{private}      &  Y               &  N                 &  N                  &  N               \\
\end{tabular}
\end{center}



Så for å kunne bruke \emph{protected} utenfor en pakke må det være en
subklasse av klassen, f.eks. \emph{class Class01 extends Package01.Class02}.
\subsection{Abstract}
\label{sec-13_4}


Vi kan ha abstrakte metoder. Hvis en subklasse arver fra en
superklasse med en abstrakt metode, er den nødt til å implementere
denne metoden. Gjøres slik:

\begin{verbatim}
abstract type methodName(parameter);
\end{verbatim}

Klassen de abstrakte metodene ligger i må da også defineres som
abstrakte, slik:

\begin{verbatim}
abstract class ClassName{ }
\end{verbatim}

Det vil ikke være mulig å lage objekter av abstrakte klasser, men kun
av subklasser som extender og implementerer alle de abstrakte
metodene. 

Abstrakte klasser gjør at vi i objektorientert programmering kan lage
en mer virkelighetslik objektmodell.
\subsubsection{Bruk av final}
\label{sec-13_4_1}


Hvis vi ønsker at en metode ikke skal kunne overskrives ved arv, kan
vi bruke \textbf{final} foran metoden. Det samme gjelder om vi ønsker en
klasse som ikke skal kunne arves, eller variabler vi ikke vil skal
kunne endres. 
\subsubsection{Forklaring}
\label{sec-14_1}
\begin{itemize}

\item class List\\
\label{sec-14_1_1}%
Her lager vi en klasse som vi sier at inneholder til den generiske typen E.

\item class Node\\
\label{sec-14_1_2}%
En intern klasse som ved å være intern overtar egenskapene til E

\end{itemize} % ends low level
\section{Generiske typer}
\label{sec-14}

Objektholdere som kan holde hva som helst, egner seg godt til å lage datastrukturer, siden strukturen blir ikke avhengig av noen spesiell type objekt å
kan derfor brukes på hva som helst. 

\begin{verbatim}
class List<E>{
    Node root;
    
    class Node{
        Node(E e){ lagretObjekt = e; } 
        E lagretObjekt;
        Node next;
    }    
    void put(E e){
        Node n = new Node(e);
        if(root == null) root = n;
        else{ 
            n.next = root; 
            root = n;
        }
    }    
    E pop(){
        E tmp = root.lagretObjekt;
        root = root.next;
        return tmp;
    }
}
\end{verbatim}

Over er et eksempel på en generisk LIFO-liste
\section{Subklasser og polyformi}
\label{sec-15}
\subsection{Arv}
\label{sec-15_1}

Vi kan arve fra en klasse i java, ved å bruke \emph{extends}. Da vil
klassen som arver bli kalt \emph{subklassen} og klassen det arves fra bli
kalt \emph{superklassen}.

Vi definerer først superklassen, som en vanlig klasse

\begin{verbatim}
class ClassName {
    int someVar = 10;
}
\end{verbatim}

Så kan vi la det som blir subklassen extende denne klassen, og arve
fra superklassen, slik:

\begin{verbatim}
class SubClass extends Classname {
    System.out.print(someVar); // Vil printe ut 10, fra superklassen.
}
\end{verbatim}

I Java kan man kun spesifisere en superklasse, men man kan godt ha et
hierarki av arv og subklasser.

Det som arv oftest benyttes til er å beskrive ett generelt objekt og
la subklasser utvide dette, så kan de igjen spesifisere de elementene
som gjør dem spesielle. F.eks. kan man tenke at man lager en
superklasse som definerer et \emph{kjøretøy}, også har du subklasser som
utvider dette, f.eks. til \emph{varebil} og \emph{personbil}.
\subsection{Tilgangskontroll og arv}
\label{sec-15_2}


Selv om du arver metoder og variabler fra en klasse, får du ikke
tilgang til elementer som er definert som \textbf{private}.
\subsection{Arv og konstruktører}
\label{sec-15_3}


Ved kjøring av en subklasse, vil både konstruktøren til subklassen og
til superklassen kjøre. Vi kan kalle på konstruktøren til superklassen
ved metoden \textbf{super(parameter)}.

Vi kan også bruke \emph{super} til å få tilgang til medlemmer av
superklassen, f.eks. \emph{super.varName} eller \emph{super.methodName()}.

Konstruktøren til superklassen vil altid eksekveres først, også går
det nedover i hierarkiet. Kaller man på super() må dette gjøres øverst
i konstruktøren.
\subsection{Polymorfi}
\label{sec-15_4}

Vi kan overskrive metoder i subklassene. Java bestemmer hvilken av
metodene som kjøres, ved run-time, ved å se på hvilke objekter som
refereres til, selv om selve pekeren er av typen Superklasse. 
\subsection{Object-klassen}
\label{sec-15_5}

Java definerer en Object-klasse, som er en superklasse til alle
klasser. Det gjør at man kan ha en referansevariabel av typen Object
som kan referere til alle typer klasser.

Object definerer ett sett med metoder man kan benytte, blant annet
\emph{Object clone()}, som kloner objektet til et nytt, selvstendig objekt.
\section{Unntak (Exceptions)}
\label{sec-16}
\subsection{Hva er unntak}
\label{sec-16_1}

Et unntak skjer når noe går galt, for eksempel å bruke en metode i et objekt
som er null(nullPointerException). Eller prøver å få tak i en verdi i et array
som er utenfor arrayets størelse (IndexOutOfBoundsException).
\subsection{Fange unntak}
\label{sec-16_2}

Å fange unntak er lett

\begin{verbatim}
try { 
    noe();
} catch (Exception e) {
    gjørDetSomSkalSkjeHvisNoeGårGalt();
}
\end{verbatim}

Man kan også bruke exceptions som en sofistikert if. Hvis man vet at koden funker, men at det
skal skje noe på et visst punkt eller etter en viss tid, kan man bruke exceptions til å
behandle unntaket når det kommer, eller bare gå videre i koden.

\begin{verbatim}
try { 
    while(noe()) { kodeSkjer(); }
} catch (Exception e) {
    løkkaErFerdigFordiNoeKultSkjedde();
}
\end{verbatim}
\subsection{Definer eget unntak}
\label{sec-16_3}

Vi kan definere våres egene unntak, hvis vi ikke finner et passende
unntak i Java sin \emph{Exception}-klasse. Det gjøres slik:

\begin{verbatim}
class EgendefinertException extends Exception {}
\end{verbatim}

Vi kan så la en klasse eller en metode bruke dette, slik:

\begin{verbatim}
void myMethod() throws EgendefinertException {
    if(a != true) {
        throw new EgendefinertException();
    }
}
\end{verbatim}

Unntaket må så kastes vidre til main, eller fanges og behandles. 
\section{Finalize, garbagecollector etc}
\label{sec-17}

finalize() er en metode du kan definere i klasser, som automatisk
kjøres før objekter blir tatt av garbage collectoren og
slettes. F.eks. hvis du skal dobbeltsjekke at en fil har blitt lukket.
\section{Tråder (Threads)}
\label{sec-18}
\subsection{Oprettelse av en tråd}
\label{sec-18_1}

Kan implemenere \emph{Runnable} eller extende \emph{Thread} classs.
Tråder er små prosesser inne i én prosess. Det er raskerkere å 
skifte fra en tråd til en annen enn fra en prosess til en annen.
Å bruke flere prosesser/tråder for å løse en oppgave kalles
paralellprogrammering.

For å oprette en tråd gjør vi følgende:

\begin{verbatim}
MyThreadClass myThreadClass = new MyThreadClass();
Thread newThread = new Thread(myThreadClass);
newThread.start(); // Kaller på run()-metoden.
\end{verbatim}
\subsection{Nyttige metoder}
\label{sec-18_2}


Hvis vi lurer på om en tråd er fullført eller ikke kan vi kalle på
metoden \emph{isAlive()}, som vil returnere en boolean.

En annen måte å vente/få beskjed når en tråd er fullført er å bruke
metoden \emph{join()}. Det gjøres på følgende måte:

\begin{verbatim}
myClass.myThread.join();
\end{verbatim}

Vi kan også sette \textbf{prioriteten} til en tråd ved å kalle metoden
\emph{setPriority(int level);}. Prioritetsnivået er et tall mellom 1
og 10. 
\subsection{Synkronisering}
\label{sec-18_3}


Vi kan enkelt synkronisere metoder, hvis vi ikke ønsker at flere
tråder skal kunne jobbe på de samme variablene, eller metodene, ved å
legge til nøkkelordet \emph{synchronized}. Den putter på en lås på metoden,
som sier at kun en tråd kan jobbe på metoden av gangen, og andre
tråder må vente.

Vi kan også synkronisere en blokk av kode, ved følgende statement:

\begin{verbatim}
synchronized(object) { 
    // Kode til å synkronisere.
}
\end{verbatim}
\subsection{Kommunikasjon mellom tråder}
\label{sec-18_4}


Hvis vi har flere tråder som skal jobbe etterhverandre på et
synkronisert objekt, men det ene objektet er avhengig av at den andre
tråden har fullført en prosedyre, kan vi benytte oss av \emph{wait()} og
\emph{notify()}-metoden. 
\subsubsection{wait()}
\label{sec-18_4_1}


Gjør at tråden sovner, og låser opp låsen på det synkroniserte
objektet. Kan kaste \emph{InterruptedException}.
\subsubsection{notify()}
\label{sec-18_4_2}


notify() vekker den andre sovende tråden, hvis den blir kalt inne i
det samme, synkroniserte objektet.
\subsubsection{notifyAll()}
\label{sec-18_4_3}


notifyAll() gjør det samme som notify(), men den vekker alle sovende tråder.
\subsection{Eksempel på en enkel tråd}
\label{sec-18_5}


\begin{verbatim}
import java.awt.*;
import java.awt.event.*;
import java.io.*;

class Klokke {
   public static void main(String[] args) throws IOException {
        System.out.println("Trykk [ENTER] for ? starte og stoppe");
   
      BufferedReader minInn = new BufferedReader
                                   (new InputStreamReader(System.in));

      minInn.readLine ( );
 
      // Her lages stoppeklokke-objektet:
      Stoppeklokke stoppeklokke = new Stoppeklokke();

      // og her settes den nye tr?den i gang.
      stoppeklokke.start();

      minInn.readLine ( );
      stoppeklokke.avslutt();
   }
}

class Stoppeklokke extends Thread {
   private volatile boolean stopp = false;
   // blir kalt opp av superklassens start-metode.
   public void run() {
      int tid = 0;
      while (!stopp) {
         System.out.println(tid++);
         try {
            Thread.sleep(1 * 1000); // ett sekund
         } catch (InterruptedException e) { }
      }
   }

   public void avslutt() {
      stopp = true;
   }
}
\end{verbatim}

Nøkkelordet \emph{volatile} brukes til å indikere at en variabel skal
endres av flere enn en tråd.
\section{Brukergrensesnitt}
\label{sec-19}


\begin{verbatim}
import javax.swing.*; 
import java.awt.*;

class RammeDemo2 extends JFrame { 
    RammeDemo2() {
        // En annen måte å sette tittel på rammen:
        super("Første vindu");                     
        setDefaultCloseOperation(JFrame.EXIT_ON_CLOSE);
        setSize(300,200);        
        setVisible(true);
    } 
    public static void main(String[] args) {
        new RammeDemo2();
    }
}
\end{verbatim}

Anonym klasse, når klassen ikke har navn, og er skrevet som parameter
til en metode.

Bruk \emph{repaint()}-metoden for å tegne vindu på nytt..
\section{Shorthands}
\label{sec-20}
\subsection{If-shorthand}
\label{sec-20_1}


\begin{verbatim}
if (condition) {
    return x;
}
return y;
\end{verbatim}

Burde isteden skrives slik:

\begin{verbatim}
return (condition ? x : y);
\end{verbatim}
\section{Prinsipper for større programmer}
\label{sec-21}

-KISS
\begin{itemize}
\item Moduler

\begin{itemize}
\item Små

\begin{itemize}
\item Definert av grensesnitt
\item Kun en funksjonalitet

\begin{itemize}
\item F.eks. LinkedList
\end{itemize}

\end{itemize}

\end{itemize}

\item Dokumentasjon
\item Testing
\item Versjonkontroll
\item Konvensjoner

\begin{itemize}
\item Hvordan kode ser ut
\item Hvordan kode struktureres
\end{itemize}

\end{itemize}


Ved over ca 40 linjer i en metode, på tide å splitte.

\end{document}
