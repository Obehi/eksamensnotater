\documentclass[english]{article}
\usepackage[T1]{fontenc}
\usepackage[latin1]{inputenc}
\usepackage{textcomp}
\usepackage{relsize}
\usepackage{babel}
\usepackage{amsmath}

\title{INF1000}
\author{Sjur Hernes}

\begin{document}

\maketitle

\section{Tall}

\begin{tabular}{|c|c|}
  \hline
  Type & Lovlige Verdier\\
  \hline
  \hline
  byte & -128 til 127\\
  \hline
  short & -32 768 til 32 767\\
  \hline
  int & -2 147 483 648 til 2 147 483 647\\
  \hline
  long & -9 223 372 036 854 775 808L til 9 223 372 036 854 775 807L\\
  \hline
  float & \textpm3.402 823 47 E+38F(6-7 signifikante desimaler)\\
  \hline
  double & \textpm1.797 693 134 862 315 70 E+308(15 signifikante desimaler)\\
  \hline
\end{tabular}

\subsection{Integere}

Integere er heltall, og lite annet.\\
\newline
Hvis vi vil konvertere en streng til en int sier vi:\\
int x = Integer.parseInt("123");\\
\newline
Hvis vi vil konvertere et annet tall til int skriver vi:\\
int x = (int) 3.14\\
\newline 
\begin{tabular}{|c|c|c|}
  \hline
  Operasjon & Beskrivelse & Eksempel\\
  \hline
  \hline
  +, -, *, /  & De fire regneartene & 1 + 1\\
  \hline
  ++ & legge til 1 & i++; , samme som i = i+1\\
  \hline
  -- & trekke fra 1 & i--; , samme som i = i-1\\
  \hline
  \% & modulo(rest) & 5 \% 2 = 1\\
  \hline
  Math.sqrt(\dots) & kvadratrot & x = Math.sqrt(4);\\
  \hline
  Math.pow(\dots) & Potens ($x^y$) & x = Math.pow(y,z);\\
  \hline
\end{tabular}
\newline
\newline
enkle presedensregler:

\begin{enumerate}
\item metodekall
\item ++ og --
\item * og /
\item + og -
\item -(som negativt fortegn)
\end{enumerate}

husk at:\\
int x, y=1;\\
x = y++ + y + ++y;\\
n� er x = 6, fordi stykket blir 1 + 2 + 3. f�rst er y = 1, og s� �ker den slik at neste gang y brukes er den 2. 
n�r vi skriver ++y s� vil y v�re 3 f�r vi legger det til.

\subsection{desimaltall}

Jeg velger � bare leke med floats, men reglene er ca det samme som for int.\\
\newline
konvertering fra heltall til desimaltall:\\
float x = (3 + 0.0) / 2;\\
alts� vi bare legger til 0.0 som det som skjer f�rst, slik at ett av tallene er et desimaltall, 
ellers vil vi f� heltallsdivisjon. Enkelt og greit, utenom det gjelder de samme opperatorene og
de samme presidensreglene.

\section{Sannhetsverdier - boolean}

\begin{tabular}{|c|c|c|}
  \hline
  Operator & Beskrivelse & Eksempel\\
  \hline
  \hline
  \&\& & Og(true hvis begge ledd er sanne) & b= (x<y) \&\& (y<z);\\
  \hline
  || & Eller(sann hvis minst ett ledd er sant) & b= (x<y) || (y<z);\\
  \hline
  ! & Ikke (snur sannhetsverdien) & b = (!true == false);\\
  \hline
  < og > & mindre enn, st�rre enn & b = x < y;\\
  \hline
  <= og >= & mendre eller lik, st�rre eller lik & b = y <= z;\\
  \hline
  == & Er lik & b= (x==y);\\
  \hline
  != & Er ikke lik & b= (x!=y);\\
  \hline
\end{tabular}
\newline
\newline
presidensregler
\begin{enumerate}
\item Metodekall
\item !
\item <, <=, >=, >
\item ==, !=
\item \&\&
\item ||
\end{enumerate}

\section{Tekst - strenger og char}

vi har strengen s = "kake"
\newline
\begin{tabular}{|c|c|c|}
  \hline
  Navn & Forklaring & Eksempel\\
  \hline
  \hline
  charAt(...) & tegnet i gitt posisjon(fra 0) & s.charAt(2)=='k'\\
  \hline
  length() & gir lengden p� teksten & s.length()==4\\
  \hline
  substring(...) & delteksten fra- og tilposisjon & s.substring(1,3)=="ak"\\
  \hline
  & gir indeksen og ut & s.substring(1)=="ake"\\
  \hline
  equals(...) & tester likhet mellom strenger & s.equals("kake")\\
  \hline
  indexOf(...) & posisjonen til tegnet/tekst & s.indexOf('a')==1\\
  \hline
  startsWith(...) & starter teksten med ... & s.startsWith("ka")\\
  \hline
  endsWith(...) & ender teksten med ... & s.endsWith("ke")\\
  \hline
  compaireTo(...) & sammenligning av tekster & s.compaireTo("bake")<0\\
  \hline
\end{tabular}

\subsection{char}

en char-verdi er rett og slett en bokstav, den kan sammenlignes ('a' < 'b') og vil da sammenlignes ut
i fra ascii-verdier (alle store bokstaver er mindre enn de sm� bokstavene).

\subsection{String}

En string er en rekke med char-verdier, alts� ord. Man kan legge ord sammen med pluss-opperatoren 
("heisann" + " " + navn), man kan konvertere tall til strenger p� denne m�ten\\
String s = "" + 42;\\
og verdien til s vil v�re "42".

Strenger kan ogs� deles opp i arrayer ved hjelp av en split-funksjon. eks:\\
string[ ] t = s.split(" ");

\section{Arrayer}

Arrayer er en indeksert(fra 0) gruppe av objekter. Man m� definere st�relsen n�r man lager objektet.\\
String[ ] a = new string[3];\\
man kan n� finne lengden p� arrayet og bruke det som en int\\
a.length;

\section{l�kker}

\begin{tabular}{|c|c|c|}
  \hline
  navn & beskrivelse & eksempel\\
  \hline
  \hline
  for & bestemt antall ganger & for(int i=0;i<3;i++)\{\}\\
  \hline
  & alle objekter i array & for(String s : a)\{\}\\
  \hline
  & alle objekter i hash & for(String s : hm.values())\\
  \hline
  while & i mens test er sann & while(1>0)\{\}\\
  \hline
  do-while & utf�rer l�kka f�r testen & do \{\} while(true);\\ 
\end{tabular}

\section{hasjkart}

Hashmaps er en enkel m�te � ordne mange objekter med et objekt som
indeks.\\
import java.util.*\\
HashMap<string,Person> personregister = new HashMap<String,Person>();\\
\newline
\begin{tabular}{|c|c|}
  \hline
  Metode & beskrivelse\\
  \hline
  \hline
  put(n�kkel, peker) & legge til objekt i HM\\
  \hline
  get(n�kkel) & hente peker til objekt\\
  \hline
  remove(n�kkel) & fjerne n�kkel fra HM\\
  \hline
  containsKey(n�kkel) & bool om n�kkelen er der\\
  \hline
  containsValue(objekt) & bool om objektet er der\\
  \hline
  values() & lager en mengde av alle verdiene i HM,\\
  & brukes til itterering\\
  \hline
  keySet() & brukes til � lage en mengde av alle n�klene\\
  & brukes til iterering\\
  \hline
  isEmpty() & returnerer true hvis tabellen er tom.\\
  \hline
  size() & Metoden returnerer antall n�kler i tabellen\\
  \hline
\end{tabular}

\end{document}
